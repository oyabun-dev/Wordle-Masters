% Exercice 1 : suite de notes (8 pts) 
% Ecrire un algorithme permettant de saisir un ensemble de n valeurs qui représentent les notes et qui 
% sont des valeurs comprises entre 0 et 20. L’algorithme doit afficher à la fin la meilleure note, la 
% moyenne des notes, le nombre de notes supérieures ou égales à 10 et le nombre de notes inferieures 
% à 10. 

% ALGORITHME: suite_de_notes
% VARIABLES: n, note, i, somme, moyenne, nb_sup_10, nb_inf_10, max
% DEBUT
%     ECRIRE "Saisir le nombre de notes"
%     LIRE n
%     somme <- 0
%     nb_sup_10 <- 0
%     nb_inf_10 <- 0
%     max <- 0

%     pour i allant de 1 à n faire
%         ECRIRE "Saisir la", i, "ème note"
%         LIRE note   
        
%         tant que note < 0 ou note > 20 faire
%         ECRIRE "Saisir une note comprise entre 0 et 20"
%         ECRIRE "Saisir la", i, "ème note"
%         LIRE note
%         fin tant que

%         somme <- somme + note
%         moyenne <- somme / n

%         si note > max alors
%             max <- note
%         fin si

%         si note >= 10 alors
%             nb_sup_10 <- nb_sup_10 + 1
%         sinon
%             nb_inf_10 <- nb_inf_10 + 1
%         fin si

%     fin pour
% FIN

% Exercice 2 : division par une suite de soustractions (6 pts) 
% Ecrire un algorithme permettant de saisir deux entiers A positif et B strictement positif l’algorithme 
% doit déterminer et afficher le quotient et le reste de la division de A sur B en utilisant une suite de 
% soustraction.

% ALGORITHME: division_par_soustraction
% VARIABLES: diviseur, dividende, quotient, reste

% DEBUT
%     ECRIRE "Saisir A"
%     LIRE diviseur
%     si diviseur < 0 alors
%         ECRIRE "Saisir A"
%         LIRE diviseur
%     fin si
%     ECRIRE "Saisir B"
%     LIRE dividende
%     si dividende <= 0 alors
%         ECRIRE "Saisir B"
%         LIRE dividende
%     fin si

%     quotient <- 0
%     reste <- diviseur

%     tant que reste >= dividende faire
%         reste <- reste - dividende
%         quotient <- quotient + 1
%     fin tant que

%     ECRIRE "Le quotient de la division de A sur B est", quotient
%     ECRIRE "Le reste de la division de A sur B est", reste
% FIN

Exercice 3 : série croissante (6 pts) 
Ecrire un algorithme qui permet de saisir une suite de réels par ordre croissant. La saisie sera arrêtée 
si est seulement si l’ordre n’est pas respecté. 

ALGORITHME: serie_croissante
VARIABLES: n, i, val_precedente, val_suivante
DEBUT
    ECRIRE "Saisir le nombre de valeurs"
    LIRE n
    ECRIRE "Saisir la première valeur"
    LIRE val_precedente
    i <- 2
    tant que i <= n faire
        ECRIRE "Saisir la", i, "ème valeur"
        LIRE val_suivante
        si val_suivante < val_precedente alors
            ECRIRE "Saisir une valeur supérieure à", val_precedente
            ECRIRE "Saisir la", i, "ème valeur"
            LIRE val_suivante
        fin si
        val_precedente <- val_suivante
        i <- i + 1
    fin tant que
FIN

